%%%%%%%%%%%%%%%%%%%%%%%%%%%%%%%%%%%%%%%%%%%%%%%%%%%%%%%%%%%%%
%% HEADER
%%%%%%%%%%%%%%%%%%%%%%%%%%%%%%%%%%%%%%%%%%%%%%%%%%%%%%%%%%%%%
\documentclass[a4paper]{article}
% Alternative Options:
%	Paper Size: a4paper / a5paper / b5paper / letterpaper / legalpaper / executivepaper
% Duplex: oneside / twoside
% Base Font Size: 10pt / 11pt / 12pt


%% Language %%%%%%%%%%%%%%%%%%%%%%%%%%%%%%%%%%%%%%%%%%%%%%%%%
\usepackage[USenglish]{babel} %francais, polish, spanish, ...
\usepackage[T1]{fontenc}
\usepackage[ansinew]{inputenc}

\usepackage{lmodern} %Type1-font for non-english texts and characters

%% Packages for Graphics & Figures %%%%%%%%%%%%%%%%%%%%%%%%%%
\usepackage{graphicx} %%For loading graphic files
%\usepackage{subfig} %%Subfigures inside a figure
%\usepackage{pst-all} %%PSTricks - not useable with pdfLaTeX

%% Please note:
%% Images can be included using \includegraphics{Dateiname}
%% resp. using the dialog in the Insert menu.
%% 
%% The mode "LaTeX => PDF" allows the following formats:
%%   .jpg  .png  .pdf  .mps
%% 
%% The modes "LaTeX => DVI", "LaTeX => PS" und "LaTeX => PS => PDF"
%% allow the following formats:
%%   .eps  .ps  .bmp  .pict  .pntg


%% Math Packages %%%%%%%%%%%%%%%%%%%%%%%%%%%%%%%%%%%%%%%%%%%%
\usepackage{amsmath}
\usepackage{amsthm}
\usepackage{amsfonts}

%

%% Line Spacing %%%%%%%%%%%%%%%%%%%%%%%%%%%%%%%%%%%%%%%%%%%%%
%\usepackage{setspace}
%\singlespacing        %% 1-spacing (default)
%\onehalfspacing       %% 1,5-spacing
%\doublespacing        %% 2-spacing


%% Other Packages %%%%%%%%%%%%%%%%%%%%%%%%%%%%%%%%%%%%%%%%%%%
%\usepackage{a4wide} %%Smaller margins = more text per page.
%\usepackage{fancyhdr} %%Fancy headings
%\usepackage{longtable} %%For tables, that exceed one page

\usepackage{url}
\usepackage[colorlinks=false, linktocpage=true]{hyperref}
\usepackage[backend=bibtex]{biblatex}
\usepackage{csquotes}
\usepackage{xargs}   
\usepackage[pdftex,dvipsnames]{xcolor}
\usepackage[colorinlistoftodos,prependcaption,textsize=tiny]{todonotes}
    
%%%%%%%%%%%%%%%%%%%%%%%%%%%%%%%%%%%%%%%%%%%%%%%%%%%%%%%%%%%%%
\hyphenation{INRIA}
%%%%%%%%%%%%%%%%%%%%%%%%%%%%%%%%%%%%%%%%%%%%%%%%%%%%%%%%%%%%%
%% Options / Modifications
%%%%%%%%%%%%%%%%%%%%%%%%%%%%%%%%%%%%%%%%%%%%%%%%%%%%%%%%%%%%%

%\input{options} %You need a file 'options.tex' for this
%% ==> TeXnicCenter supplies some possible option files
%% ==> with its templates (File | New from Template...).
%
\newcommand{\DIPISASWHinitiative}{SWH@DIPISA }
\newcommand{\TAUMUS}{TAUMUS }
\renewcommand{\th}{$^{th}$ }
\newcommand{\sourcecode}{SC}

%%TODOO

\newcommandx{\guido}[2][1=]{\todo[linecolor=red,backgroundcolor=red!25,bordercolor=red,#1]{#2}}
\newcommandx{\carlo}[2][1=]{\todo[linecolor=blue, backgroundcolor=blue!25,bordercolor=blue,#1]{#2}}
\newcommandx{\laura}[2][1=]{\todo[linecolor=OliveGreen,backgroundcolor=OliveGreen!25,bordercolor=OliveGreen,#1]{#2}}
\newcommandx{\roberto}[2][1=]{\todo[linecolor=Plum,backgroundcolor=Plum!25,bordercolor=Plum,#1]{#2}}
\newcommandx{\nascosto}[2][1=]{\todo[disable,#1]{#2}}
%
\bibliography{ExtAbstrSWH}
%
\title{Saving the Software Heritage:
				\\ the Process  
				\\ 	\large Extended Abstract
}
\author{
    Laura Bussi\\
    Dept. of Computer Science\\
    University of Pisa\\
    l.bussi1@studenti.unipi.it
	\and
    Roberto Di Cosmo\\
    Software Heritage, Inria and University of Paris\\
    roberto@dicosmo.org
  \and
    Carlo Montangero\\
    Dept. of Computer Science\\
    University of Pisa\\
    carlo@montangero.eu
  \and
    Guido Scatena\\
		Dept. of Computer Science\\
    University of Pisa\\
    guido.scatena@unipi.it
}
\date{\today}



\begin{document}
\maketitle

%\guido{mancano dettagli degli autori di Laura}

%{ten paragraphs, 100 words each, ten lines each in this format on the average}
\noindent              
Software is everywhere,  binding our personal and social lives, embodying a vast part of the technological knowledge that powers our industry, supports modern research, mediates access to digital content and fuels innovation. In a word, a rapidly increasing part of our collective knowledge is embodied in, or depends on software artifacts. 

Software does not come out of the blue: it is written by humans, in the form of software Source Code (SC), is a precious, unique form of knowledge that, besides being readily translated into machine executable form, should also ``be written for humans to read'' \cite{AbelsonS85}, and ``provides a view into the mind of the designer'' \cite{Shustek06}. It is essential to preserve this precious technical, scientific and cultural heritage over the long term.\\

\noindent
Software Heritage is a non profit, multi-stakeholder initiative, launched by {INRIA} in partnership with {UNESCO}, that has taken over this challenge. Its stated mission is to collect, preserve, and make readily accessible all the software source code ever written, in the Software Heritage \emph{Archive}.\\

Software Heritage designed specific strategies to collect software according to its nature~\cite{swhcacm2018}. 

For software that is easily accessible online, and that can be copied without specific legal authorizations, the approach is based on automation. This way, as of May 2019, Software Heritage has already archived almost 6 billion unique SC files from over 89 million different origins, focusing in priority on popular software development platforms like GitHub and GitLab and rescuing SC from Google Code and Gitorious that hosted more than 1.5 million projects, and now shutdown.

For source code that is not easily accessible online or requires curation, a different approach is needed, based on a focused search, and with significant human intervention: the Software Heritage Acquisition Process (SWHAP) to \emph{rescue, curate and illustrate} landmark legacy SC. SWAP needs to cope with the variety of physical media where the source code may be stored, the multiple copies and versions that may be available, the potential input of the authors that are still alive, and the existence of ancillary material like documentation, articles, books, technical reports, email exchanges.\\

\noindent
This contribution presents the approach we are taking with the definition, pilot implementation and experimental usage of SWHAP, in order to get feedback from the HaPoC community on our goals and methods. 

\paragraph{Requirements}
First of all, we identified the following high-level requirements, for the activities of \emph{preservation, curation and illustration}.

% They should accommodate all the furnishers, be them the scavengers searching around for relevant SC or the authors conferring their own work to SWH.

% Coverage of both offline and online SC is also required.

\begin{description}
	\item[Preservation] of the artifacts is a primary concern: we need a place where to save the original raw material that will be later studied and manipulated. Depending on the nature of the raw material, these places may be physical or virtual, as follows:
	\begin{description}
		\item[Warehouse]: a physical location where physical raw material is safely archived and stored, with the usual acquisition process (for example, see~\cite{Spectrum})
		\item[Depository]: a virtual space where is safely archived digital raw materials, either obtained directly in digital form, or extracted from physical supports; a proper acquisition process must be put in place to ensure that this material is traceable.
        \end{description}
      \item[Curation] of the raw digital material found in the Depository will produce clean and historified \sourcecode~that can be ingested in the SWH Archive, and ancillary material that may be deposited in locations like WikiData.
        \begin{description}
          \item[Workbench] The curation process for source code can be complex, as can be seen in the great example of the history of Unix~\cite{Spi16g}, and we recommend to have a standard environment where it can be carried out, with support for logging the various operations.
        \end{description}
      \item[Illustration] of the history of relevant source code may require access to all of the above locations: the Warehouse, the Depository, the Workbench, the Software Heritage Archive, WikiData, as well as other external resources.
\end{description}

We also agreed on the following supporting principles: 

\begin{description}
	\item[Convergence] The activities related to offline (legacy) SC should converge as early as possible to those related to online SC.
	\item[Openness] Any supporting implementation should be 
	based on open and free tools and standards.
	\item[Interoperability] Any supporting implementation should provide support for 
the cooperation and coordination of the many actors playing the many roles of the acquisition process.	
\end{description}

\noindent
We are currently experimenting with a particular implementation of the process where the two virtual locations, Depository and Workbench, use GitHub as a common technological backbone.

GitHub is a platform that offers not only a Version Control System (VCS), extremely
useful for tracking various version of digital artifacts, but also a wealth of
tools to facilitate collaboration among a variety of users that seems very well
suited for supporting the multi-users acquisition process we aim at.

Moreover, the public part of GitHub is already automatically archived by Software Heritage, which
means that all the artifacts stored in the Depository and the steps performed in
the Workbench will be safely archived alongside the results of the curation process: we believe this will facilitate convergence.

To facilitate the access to the artifacts, we need to adopt a standard ontology, and for this we are evaluating the CodeMeta Project \cite{CodeMeta}, that provides a standard vocabulary for describing software artifacts.

\noindent
To archive the ancillary documents, Wikimedia \cite{wiki:Wikimedia_movement} is a natural choice, given its openness and freeness.  

Over the next months, we will run the process on a representative sample of SC
developed in Pisa,  including i) a representative offline digital software of the '90s, the CMM system \cite{CMM:1994}, developed using a in house VCS, and ii) a representative offline software of the '70s, available only as paper listing, the TAUMUS system for music generation \cite{TAU2:1976}.

We hope to demonstrate the feasibility of a process that, leveraging the
powerful infrastructures available today, ranging from Software Heritage to
Wikimedia, through GitHub, will be easy to be adopted worldwide by all the
passionate people and institutions that want to contribute and rebuild 
the history of computer programming.

%\noindent
%10 - References 


%\nocite{*}
\printbibliography

%\listoftodos[Notes and ToDo]


% 1     Preface to Abelson, Sussman, and Sussman, The Structure and     Interpretation of Computer Programs, MIT Press, 1985
	
%  2     Free Software Foundation, Inc., The GNU General Public License,     Version 3, 1, 2007
	
%  3     Shustek, L. J. What Should We Collect to Preserve the History of     Software?, IEEE Annals of the History of Computing, 2006
\end{document}

% from latex to md
% pandoc -s ExtAbstrSWH.tex -o ExtAbstrSWH.md
